\chapter{Predictive Models}
Predictive models help us summarize and understand data in order to
\begin{inparaenum}[1)] \item
make predictions about the future, and/or \item explain interesting
relationships in the data. 
\end{inparaenum}
The ultimate purpose of a model can be for prediction or explanation,
but we usually want some combination of both. In this chapter, the
term ``predictive model'' includes the explanatory aspects.
Predictive models are broadly categorized into two types: regression
and classification.  The difference is in the nature of the response
variable, i.e.  the ``thing'' that we are trying to predict or
explain.  Regression indicates a numeric response, e.g. a stock price,
population growth, or sales of a product. Classification means that
the response is categorical, e.g. gender, category of an email, or
presence/absence of a disease.

Regardless of whether the model is one of regression or classification, 
modeling is a process that almost always involves iterating
over the following steps~\cite{chambers:1992}.
\begin{compactenum}
\item obtaining data \label{s1}
\item choosing a candidate model  \label{s2}
\item fitting the model, i.e. using software to  estimate the model parameters \label{s3}
\item interpretation of the fitted model parameters \label{s4}
\item diagnostics to see in what ways the model \emph{fails} to fit the data \label{s5}
\end{compactenum}
The wide availability of high-quality, open-source software for
statistical computing has made step~\ref{s3} the easiest part. In this
chapter, we concentrate on steps~\ref{s2}, \ref{s4}, and \ref{s5}.

\section{Regression}

\subsubsection*{Model Formulas}

Regression (and classification) models are simplified descriptions of
data that involve mathematical relationships. In addition to the data
itself, a model consists of \begin{inparaenum}[1)] \item a
  formula that specifies the mathematical relationships among the
  variables, and \item a description of how well the data agree with
  the model.
  \end{inparaenum} 
  Let's start with an example of how model formulas are represented in
  the R programming language. Recall the data set from
  Chapter~\ref{data-analysis} on college students and driving speed.
  This data contains \num{1325} observations on gender, height, and
  the fastest speed ever driven (in mph) for a sample of college
  students.  Do we really think that, on average, males drive faster
  than females (or vice versa)? Similarly, is there any relationship
  between height and speed? Consider the following model formula
written in R.
\begin{Verbatim}
speed ~ height + gender
\end{Verbatim}
You should read the formula above as ``Speed is modeled as a linear
function of height and gender''.  The term to the left of the
``\mtilde'' is the response (a.k.a dependent variable, output
variable) and the terms to the right of the ``\mtilde'' that are
separated by a ``\texttt{+}'' are the predictor variables (a.k.a
explanatory variables, independent variables, features, input
variables).  Mathematically, the formula specifies the following
model.
\begin{equation}
  speed_i = \beta_0 + \beta_1 height_i + \beta_2 gender_i + \epsilon_i, \qquad i=1,\ldots,n
\label{eq:speed-height-gender}
\end{equation}

The coefficients $\beta_0,~\beta_1,~\beta_2$ along with the variance
$\sigma_{\epsilon}^2$ of the error term $\epsilon$ are the \emph{parameters}
to be estimated from the data.  The index $i$ refers to the $i$th
observation in the data set (in this case, the $i$th person). Notice
that neither the coefficients nor the error term appear in the R
formula.  They are inferred and so we do not need to enter them. In
particular the R formula does not include the intercept term
$\beta_0$. It is there by default; this is usually what we want. The R
formula above is equivalent to
\begin{Verbatim}
speed ~ 1 + height + gender
\end{Verbatim}
If we do not want an intercept term in our model, perhaps because we want to
force the fit to go through the origin, then we need to explicitly remove
the intercept, like this:
\begin{Verbatim}
speed ~ -1 + height + gender
\end{Verbatim}

``Fitting'' a model means to obtain estimated values for the
parameters (i.e. the coefficients) along with an indication of how
well the data agrees with the fitted parameters. The data set on
gender, height, and speed is from a sample of \num{1325} students.  If
we obtained the same data on a different sample of students (but drawn
from the population of college students with similar attributes), we
would expect the fitted parameter values from the second model to be
close to, but not exactly the same as, those from the original
sample. Imagine obtaining such a data set and fitting the model many
times.  For each coefficient, you would obtain a distribution of
values. Usually, we are interested in whether this distribution of
values could plausibly contain the value zero. If it does, then we say
that there is no statistical relationship between the associated
predictor variable and the response.

\subsubsection*{Understanding Regression Output}

When fitting a model in the R programming language, the paradigm is to
save the fitted model as an object. Then, we can extract explanatory
information from the object and use that information to make
inferences or predictions on future observations. For simplicity,
let's ignore gender and fit a regression model with speed as the
response and height as the only predictor.

\begin{Verbatim}[numbers=left,xleftmargin=5mm]
Speed <- read.csv("../data/speed-gender-height.csv")

fm <- lm(speed ~ height, data = Speed)
summary(fm)
\end{Verbatim}

The call to \texttt{lm()} on line 3 fits the model and stores the
information about the fir in an object named \texttt{fm}\footnote{You
  can name the object anything you like. I tend to use \texttt{fm} for
  ``fitted model''. It's nice and short.}.  We then ask for summary
information about the fit. The output from line 4 is shown in
Figure~\ref{fig:regoutput1}.

\begin{SaveVerbatim}{regoutput1}
> 
Call:
lm(formula = speed ~ height, data = Speed)

Residuals:
    Min      1Q  Median      3Q     Max 
-98.859  -8.248   1.673  11.635  81.992 

Coefficients:
            Estimate Std. Error t value Pr(>|t|)    
(Intercept)  -0.7145     9.8671  -0.072    0.942    
height        1.3830     0.1489   9.287   <2e-16 ***
---
Signif. codes:  0 ‘***’ 0.001 ‘**’ 0.01 ‘*’ 0.05 ‘.’ 0.1 ‘ ’ 1

Residual standard error: 21.75 on 1300 degrees of freedom
  (23 observations deleted due to missingness)
Multiple R-squared:  0.06222,	Adjusted R-squared:  0.0615 
F-statistic: 86.25 on 1 and 1300 DF,  p-value: < 2.2e-16
\end{SaveVerbatim}

\begin{figure}
\fbox{
\begin{minipage}{\textwidth}
\BUseVerbatim{regoutput1}
\caption{Summary regression output from R for the model with height
  as the only predictor variable.}
\label{fig:regoutput1}
\end{minipage}
}
\end{figure}

The model formula is repeated in the summary output. Recall that, mathematically,
this model is
\begin{equation}
 speed_i = \beta_0 + \beta_1 height_i + \epsilon_i    \label{eq:reg1}
\end{equation}
 where
$\epsilon_i$ is the error term, i.e. the difference between the
model's fitted value of speed for observation $i$ and the actual value
of speed for observation $i$.  This is also known as the residual. One
assumption about linear regression models is that the residuals are
Normally distributed with mean zero and some variance
$\sigma_{\epsilon}^2$.
\begin{equation}
 \epsilon_i \sim \mathcal{N} \left( 0, \sigma_{\epsilon}^2 \right) 
\label{eq:resid}
\end{equation}
The output section entitled \texttt{Residuals:} provides a rough idea
of this distribution. If the residuals are distributed according
to~\ref{eq:resid}, then we would expect the distribution to be
symmetric, with median equal to zero and with first and third
quartiles that are the same distance from zero. Of course, we are
working with a sample of data, so we cannot expect perfect symmetry.
The residuals for this particular model appear to be symmetric, or at
least not too far from it.

Model \ref{eq:reg1} has three parameters: the intercept $\beta_0$, the
coefficient on height $\beta_1$ (a.k.a the slope), and the variance of
the residuals $\sigma_{\epsilon}^2$.  All three parameters are
estimated from the data. The output section entitled
\texttt{Coefficients:} provides a table of information about the
fitted coefficients. The table has four columns:

\vspace{.1in}
\begin{tabular}{ll}
\texttt{Estimate}& the estimated value of the coefficient\\
\texttt{Std. Error}& the uncertainty in the estimated value\\
\texttt{t value} & the test statistic in the test for whether $\beta_j=0$\\
\texttt{Pr(>|t|)}& the associated $p$-value\\
\end{tabular}
\vspace{.1in}

Let's take the coefficient on height as an example. From the table,
the estimate is $\hat{\beta}_1 = 1.383$. A regression coefficient is a 
statistic; it is a summary of data. Because a regression coefficient
can be written as the sum of random variables, the Central Limit Theorem
tells us that $\hat{\beta}_1$ is Normally distributed with mean equal
to its true value and with variance equal to $\sigma^2_{\beta_1}$, or equivalently
with standard deviation equal to $\sigma_{\beta_1}$. The standard deviation
of a statistic is called the standard error, and from the table we see
that $\hat{\sigma}_{\beta_1} = 0.1489$.

Recall that we are usually interested in whether $\beta_1=0$,
i.e. whether there is a relationship between speed and height. The
formal ``hypothesis test'' is
\begin{align}
\begin{split}
  H_0 &: \beta_1 = 0 \\
  H_A &: \beta_1 \neq 0
\label{eq:hyp}
\end{split}
\end{align}
where $H_0$ is called the null hypothesis, which is sort of like the default
position. $H_A$ is called the alternative hypothesis. The question is whether
the data provides enough evidence to reject the null hypothesis in favor
of the alternative hypothesis. The test is set up so that if you reject
the null hypothesis, then you have found something interesting to report
(e.g. ``On average, taller people drive faster!'') The third column of the
summary output reports the standardized value of $\beta_1$ when the null
hypothesis is true. It is the test statistic for the hypothesis test
\ref{eq:hyp}.
\[ t_0 = \frac{\hat{\beta}_1 - 0}{\hat{\sigma}_{\beta_1}} =
  \frac{1.383 - 0}{0.1489} = 9.288 \] If the null hypothesis is true,
the test statistic $t_0$ has a $t$ distribution with degrees of
freedom equal to the number of observations in the data set minus the
number of estimated coefficients.  A value of 9.288 falls very far into
the right tail of the $t$ distribution.  The area to the right of
9.288 is called the $p$-value. It represents the probability that null
hypothesis really is true, given this data set. This value is reported
in the last column under \texttt{Pr(>|t|)}. In our case, the $p$-value
for $\beta_1$ is extremely small, less than $2 \times 10^{-16}$.  What
we conclude from all of this is that the data provides strong evidence
of a statistical relationship between speed and height.

The numeric interpretation of the value $\hat{\beta}_1 = 1.383 \neq 0$
is that, on average, a one-unit increase in height corresponds to a
1.383-unit increase in speed.  Here, we need to pay attention to the
units in the data. Speed is reported in miles per hour and height is
reported in inches. So, the data indicate that for each one inch
increase in height, the fastest speed ever driven increases by 1.383
miles per hour (again, this is ``on average''). Although the
coefficient on height is statistically significant, is it
\emph{practically} significant? Let's say that 3 inches represents a
practical difference in height. Then our model tells us that (on
average) the difference in speed is a $3 \times 1.383 \approx 4$ miles
per hour.  The average ``fastest speed'' is about 91 miles per hour.
So, while the effect is statistically significant, the magnitude of
the effect is mediocre at best.

The estimated value for $\sigma_{\epsilon}$ is provided in the
last section of the output. We see that $\hat{\sigma}_{\epsilon} = 21.75$.
It is reported as the residual standard error (i.e. standard deviation), 
rather than as the variance
of the residuals. Notice that R reports that this estimate is based on
\num{1300} ``degrees of freedom''. We can think of the degrees of freedom
as the effective size of the data set. In this case, we have \num{1325}
observations in the data set, but 23 observations contain either a missing
value for speed or a missing value for height. Also, we lose a 
degree of freedom for each estimated coefficient. This leaves
$\num{1325} - 23 - 2 = \num{1300}$ degrees of freedom.

For completeness, the last two lines of the output show the values for
$R^2$ and the $F$-statistic for the overall hypothesis test of whether
there is a relationship between any of the predictor variables and the
response.  Model~\ref{eq:reg1} has a single predictor variable. In
this case, the overall hypothesis test is equivalent to the hypothesis
test~\ref{eq:hyp}\footnote{In the case of a single predictor
  variable, the $F$-statistic is equal to the squared value of the
  $t$-statistic ($86.246 = 9.287^2$).  Also notice that the $p$-value
  for the overall test matches the $p$-value for the coefficient on
  height, the only predictor variable.}.  We can think of $R^2$ as the
proportion of variation in the data that is explained by the
model. To be specific, if the actual response values are $y_i$ and the fitted
values from the model are $\hat{y}_i$, where $i=1,\ldots,n$, 
then $R^2$ is computed as follows.
\begin{align}
\begin{split}
TSS =& \sum_{i=1}^n \left( y_i - \bar{y} \right)^2\\
RSS =& \sum_{i=1}^n \left( y_i - \hat{y}_i \right)^2\\
R^2 =& 1 - RSS/TSS
\end{split}
\label{eq:R2}
\end{align}
where $\bar{y}$ is the sample mean of the response. $TSS$ represents
the total sum of squares around the sample mean. It is the total
variation in the response (if we were to simply use the sample mean to 
make predictions). $RSS$ represents the sum of squares around the
regression line.  The ratio $RSS/TSS$ is the proportion of (squared)
variation that is \emph{unexplained} by the model.

The reported value of $R^2 = 0.06222$ seems low, but this typical of
real-world data sets. Figure~\ref{fig:speed-height} shows the data
with the regression line from model~\ref{eq:reg1} over-layed.  There
is a detectable slope in the data, but there is also a lot of
variation around the regression line. The code to produce
Figure~\ref{fig:speed-height} is

\begin{Verbatim}[numbers=left,xleftmargin=5mm]
ggplot(Speed, aes(x=height, y=speed)) +
    geom_point() +
    geom_smooth(method="lm", se=FALSE)
\end{Verbatim}


\begin{figure}
\begin{center}
\includegraphics[width=0.7\columnwidth]{speed-height.pdf}
\caption{The fitted model \texttt{speed~\mtilde~height}. Height is the 
single predictor variable. $R^2 \approx 0.062$.}
\label{fig:speed-height}
\end{center}
\end{figure}

Keep in mind that $R^2$ is not an indication of model
``correctness''. Rather, it is a measure of the tightness of the
data to the fitted model. Indeed, models with high values of $R^2$ are very
often not all that interesting. Figure~\ref{fig:mpg-wt} shows the data
and over-layed regression line for a model from the often-cited
\texttt{mtcars} data set that is available in R. This data set
contains 32 observations on various attributes of automobiles. The
regression model with \texttt{mpg} (miles per gallon) as the response
and \texttt{wt} (vehicle weight) as the only predictor variable
provides an $R^2 \approx 0.75$, which is quite large for a regression
model. The model may be useful when designing an automobile, but the
result is unsurprising.

\begin{figure}
\begin{center}
\includegraphics[width=0.7\columnwidth]{mpg-wt.pdf}
\caption{The model \texttt{mpg~\mtilde~wt} from the \texttt{mtcars} data set. 
$R^2 \approx 0.75$.}
\label{fig:mpg-wt}
\end{center}
\end{figure}

\subsubsection*{Regression with a Categorical Predictor Variable}

Is it possible to improve the fit of the model to predict fastest
speed ever driven (and perhaps increase $R^2$)? Typically, two options
are available to address this question:
\begin{inparaenum}[1)] \item more (and/or better) predictor variables,
  and \item a different model.\end{inparaenum}. The latter option is
appropriate when, for example, we observe a nonlinear trend in the
data. We would then want to create a model that could capture the
nonlinear relationship, e.g., a nonlinear regression model, a
tree-based model, or a neural network. Regarding the first option, the
data set contains one other variable: \texttt{gender}. Let's add it as
the second predictor variable.

\begin{Verbatim}[numbers=left,xleftmargin=5mm,samepage=true]
fm2 <- lm(speed ~ height + gender, data = Speed, na.action="na.exclude")
summary(fm2)
\end{Verbatim}

Two changes are made to the call to \texttt{lm()}:
\begin{inparaenum}[1)]
\item we add gender to the model formula, and
\item we explicitly exclude observations with missing values.
\end{inparaenum}
The output from the call to \texttt{summary()} is shown in
Figure~\ref{fig:regoutput2}. The residuals still appear to be
(roughly) symmetric about zero. Notice that the table of information
on the coefficients now has a row for $\beta_2$, the coefficient on
gender, and that the label for the coefficient is \texttt{gendermale}.
Recall that gender is a categorical predictor variable. In R, the categories
(i.e. male and female) are called \emph{levels}.

\begin{Verbatim}[samepage=true]
> levels(Speed$gender)
[1] "female" "male"  
\end{Verbatim}

The levels are simply labels; however, the underlying mathematical
model knows nothing about the labels.  In model
\ref{eq:speed-height-gender}, which is repeated here, the categorical
variable
$gender_i$ is an indicator variable that takes the value 0 or 1.
\begin{equation*}
  speed_i = \beta_0 + \beta_1 height_i + \beta_2 gender_i + \epsilon_i, \qquad i=1,\ldots,n
\tag{\ref{eq:speed-height-gender} revisited}
\end{equation*}
where $height_i$ is a numeric value in units of inches and
\begin{equation*}
  gender_i = \begin{cases}
0 \quad \text{if person $i$ is female,}\\
1 \quad \text{if person $i$ is male.}
\end{cases}
\end{equation*}

For categorical variables, one level is called the ``reference
level''. The effect of all other levels on the response are with
respect to the reference level. In this case, the reference level is
\texttt{female}. You can think of the variable $gender_i$ as a switch
that gets flipped on if person $i$ is male. When $gender_i = 1$, the
estimated value for $\beta_2$ is added to the response. In this way,
we can interpret $\hat{\beta}_2 = 5.7276$ as the average difference
between males and females for the fastest speed ever driven. Now, we
are in a position to understand why R concatenates the non-reference
level to the identifier for the coefficient for gender
(i.e. \texttt{gendermale}).  The mnemonic is ``add 5.7276 to the
predicted response if observation $i$ has \texttt{male} as the level
for gender''.  For categorical variables with more than two levels,
the \texttt{Coefficients:} table will contain an entry for each level
except the reference level.

\begin{SaveVerbatim}{regoutput2}
>
Call:
lm(formula = speed ~ height + gender, data = Speed, na.action = "na.exclude")

Residuals:
     Min       1Q   Median       3Q      Max 
-100.234   -7.717    2.283   11.313   81.857 

Coefficients:
            Estimate Std. Error t value Pr(>|t|)    
(Intercept)  24.6755    12.1534   2.030 0.042525 *  
height        0.9699     0.1885   5.145 3.09e-07 ***
gendermale    5.7276     1.6142   3.548 0.000402 ***
---
Signif. codes:  0 ‘***’ 0.001 ‘**’ 0.01 ‘*’ 0.05 ‘.’ 0.1 ‘ ’ 1

Residual standard error: 21.66 on 1299 degrees of freedom
  (23 observations deleted due to missingness)
Multiple R-squared:  0.07122,	Adjusted R-squared:  0.06979 
F-statistic:  49.8 on 2 and 1299 DF,  p-value: < 2.2e-16
\end{SaveVerbatim}

\begin{figure}
\fbox{
\begin{minipage}{\textwidth}
\BUseVerbatim{regoutput2}
\caption{Summary regression output from R for the model with both
height and gender as predictor variables.}
\label{fig:regoutput2}
\end{minipage}
}
\end{figure}

In effect, the model~\ref{eq:speed-height-gender} contains two
regression lines, one for males and one for females. The effect of
height on speed (the slope) is assumed to be the same for both
genders.  A plot of the fitted regression model is shown in
Figure~\ref{fig:speed-height-gender}. The separation between the lines
for male and female is the value $\hat{\beta}_2 = 5.7276$. If we have
reason to believe that the effect of height on speed varies by gender,
then we would add an interaction term to the model. The addition of
interaction term may be warranted, but doing so complicates the
interpretation of the model. This modification to the model is
explored in Exercise~\ref{ex:interaction}.

\begin{figure}
\begin{center}
\includegraphics[width=0.7\columnwidth]{speed-height-gender.pdf}
\caption{The fitted model \texttt{speed~\mtilde~height + gender}. $R^2 \approx 0.071$.}
\label{fig:speed-height-gender}
\end{center}
\end{figure}

The code to produce Figure~\ref{fig:speed-height-gender} is shown
below.  On line 1, we add the predicted (i.e. fitted) values from the
model to the \texttt{Speed} data frame so that we can overlay the
fitted regression lines for males and for females.  If we did not use
the predicted values from the call to \texttt{predict()}, then the
call to \texttt{geom\_smooth} would use height alone as the predictor
(i.e. it would not include gender).  This is a work-around so that we
can show the fit from the full model.

\begin{Verbatim}[numbers=left,xleftmargin=5mm,samepage=true]
Speed <- mutate(Speed, fit = predict(fm2, na.action=NULL))

ggplot(Speed, aes(x=height, y=speed, color=gender)) +
    geom_point() +
    geom_smooth(method="lm", mapping=aes(y=fit))
\end{Verbatim}


\subsubsection*{Transformations on data.}

Sometimes we can transform the data in order to model nonlinear
relationships among variables, but in the framework of a linear model.
With experience, you will be able to recognize situations when this
technique may be appropriate. As an example, the \texttt{msleep} data
set is available in R as part of the \texttt{ggplot2} package.  This
data set contains, among other things, the total hours of sleep per
day for 83 species of mammals. To read about the data set, type
\begin{Verbatim}
> ?msleep
\end{Verbatim}
at the R prompt. Suppose that an animal sciences researcher would like
to explain the amount of sleep as function of brain weight. First, we
plot total sleep in hours versus brain weight in kilograms.

\begin{Verbatim}[samepage=true]
ggplot(msleep) +
    geom_point(aes(x = brainwt, y = sleep_total))
\end{Verbatim}

The plot is shown in Figure~\ref{fig:sleep-brainwt}. At first glance, the
data does not appear promising for a linear model; however, just
looking at the spread of the brain weight values along the $x$ axis,
we notice that the small values are ``bunched up'' and the larger
values are spread out.  This is an indication that a transformation
may be appropriate. For mathematical (and biological) reasons, a
logarithmic transformation is a common approach to make the data more amenable
to the assumptions required by a linear regression model.  By taking the
logarithms of the brain weights, we penalize (or shrink) the large values
toward the smaller values (on the log scale). 

\begin{Verbatim}[samepage=true]
ggplot(msleep, aes(x = log(brainwt), y = sleep_total)) +
    geom_point() +
    geom_smooth(method="lm", se=FALSE)
\end{Verbatim}

\begin{figure}
\begin{center}
\includegraphics[width=0.7\columnwidth]{sleep-brainwt.pdf}
\caption{Total sleep in hours vs. brain weight in kg from the \texttt{msleep}
data set.}
\label{fig:sleep-brainwt}
\end{center}
\end{figure}

Figure~\ref{fig:sleep-log-brainwt} shows total sleep versus the
log-transformed brain weight along with a fitted regression line
\footnote{When taking a logarithmic transforming of a predictor
  variable, it's common to use the natural logarithm, but any
  logarithmic transformation will have a similar effect.}. The pattern
is now clear: on average, mammals with larger brains sleep
less. Moreover, the residuals appear to be roughly symmetric about the
regression line. For simplicity and reliability during the fitting
process, we would like to use linear models whenever possible;
however, the price we pay for transforming a predictor variable is
increased complexity when we interpret the estimated parameters.

\begin{figure}
\begin{center}
\includegraphics[width=0.7\columnwidth]{sleep-log-brainwt.pdf}
\caption{Total sleep in hours vs. logarithm of brain weight in kg from the \texttt{msleep}
data set.}
\label{fig:sleep-log-brainwt}
\end{center}
\end{figure}

Fitting the model in R is straightforward.
\begin{Verbatim}[samepage=true]
fm <- lm(sleep_total ~ log(brainwt), data=msleep)
summary(fm)
\end{Verbatim}
Mathematically, this model is
\begin{equation}
  \label{eq:sleep-log-brainwt}
  sleep\_total_i = \beta_0 + \beta_1 \ln brainwt_i + \epsilon_i
\end{equation}

The summary output is shown in Figure~\ref{fig:regoutput3}.  An
understandable interpretation of $\hat{\beta}_1$, the coefficient on
brain weight, is more difficult, but the basic linear relationship
remains the same. A one-unit increase in the logarithm of brain weight
corresponds to a decrease in total sleep time of about on hour (on
average). Perhaps the easiest way to get an idea of the effect of
brain weight on sleep time is to plug in some values and observe the
difference. Suppose we are interested in comparing the difference in the 
amount of sleep for two mammals, one with a brain weight of 2 kg
and one with a brain weight of 3 kg. The average difference in sleep
time is about 0.421 hours, or about 25 minutes.
\begin{Verbatim}[samepage=true]
> (5.9474 -1.039*log(2)) - (5.9474 -1.039*log(3))
[1] 0.4212782
\end{Verbatim}


\begin{SaveVerbatim}{regoutput3}
Call:
lm(formula = sleep_total ~ log(brainwt), data = msleep)

Residuals:
    Min      1Q  Median      3Q     Max 
-6.0734 -2.8483 -0.6479  2.4049  9.5395 

Coefficients:
             Estimate Std. Error t value Pr(>|t|)    
(Intercept)    5.9474     0.9135   6.510 2.57e-08 ***
log(brainwt)  -1.0397     0.1914  -5.432 1.36e-06 ***
---
Signif. codes:  0 ‘***’ 0.001 ‘**’ 0.01 ‘*’ 0.05 ‘.’ 0.1 ‘ ’ 1

Residual standard error: 3.588 on 54 degrees of freedom
  (27 observations deleted due to missingness)
Multiple R-squared:  0.3534,	Adjusted R-squared:  0.3414 
F-statistic: 29.51 on 1 and 54 DF,  p-value: 1.361e-06
\end{SaveVerbatim}

\begin{figure}
\fbox{
\begin{minipage}{\textwidth}
\BUseVerbatim{regoutput3}
\caption{Summary regression output from R for mammalian sleep in hours
with log-transformed brain weight as the predictor variable.}
\label{fig:regoutput3}
\end{minipage}
}
\end{figure}


\section{Classification}

\section{Time Series}

\section{Exercises}

\begin{enumerate}

\subsubsection*{Regression}
  
% this problem is OK
\item \emph{Old Faithful.}
In the \texttt{datasets} package in R there is a data set named
\text{faithful} that
contains data on the Old Faithful geyser in Yellowstone National Park,
Wyoming, USA.  The variables in the data set are
\begin{compactitem}[$\circ$]
\item \texttt{eruptions} the eruption time in minutes
\item \texttt{waiting} the time in minutes until the next eruption
\end{compactitem}

To view information about the data set, type
\begin{Verbatim}
> ?faithful
\end{Verbatim}
at the R prompt. Use this data to perform the following exercises.

\begin{enumerate}
\item Create histograms of \texttt{eruptions} and \texttt{waiting}
  (separately). \label{hist}

\item Create a scatter plot (i.e. an x--y plot) with \texttt{eruptions}
  on the x axis and \texttt{waiting} on the y axis. \label{scat}

\item From the histograms and the scatter plot that you created, what
  can you say about the behavior of the Old Faithful geyser? \label{interp}
  
\item Fit a linear regression model using the \texttt{lm()} function
  with \texttt{waiting} as the response variable and \texttt{eruptions}
  as the only predictor variable. Print a summary of the results.
  \begin{enumerate}
  \item What is the interpretation of the intercept? \label{int}
  \item What is your interpretation of the fitted coefficient
    on \texttt{eruptions}? \label{dur}

  \item You just observed an eruption of duration 4 minutes.
    Make a prediction on how long you will have to wait until
    the next eruption. Can you make any statement about the
    uncertainty in your prediction? In other words, can
    you give a range for the time until the next eruption?
    Don't worry about being exact with your range, just
    give something reasonable. \label{pred}.
  \end{enumerate}
\end{enumerate}

% re-written by Braeden 
\item \emph{Cherry trees.} In the \texttt{datasets} package in R, the
  \texttt{trees} data set contains measurements on the girth
  (diameter) in inches, height in feet, and volume in cubic feet of 31
  black cherry trees.
\begin{enumerate}
\item Fit a regression model with volume as the response
variable and girth as the predictor variable.
\item Plot the data and overlay the fitted regression line.
\item Provide an interpretation for the coefficient on girth.
\item In your own words, state your interpretation of the $p$-value
for the coefficient on girth.
\end{enumerate}
  
% written by braeden
\item \emph{Fuel efficiency.}
  For this problem we will be using a dataset called \texttt{mtcars}
  from the \texttt{datasets} package in R. This dataset contains data
  about different types of cars.  Fit a linear regression model using
  \texttt{lm()} with miles per gallon (\texttt{mpg}) as the response
  variable and the following predictor variables:
  \begin{compactitem}
  \item number of cylinders (\texttt{cyl})
  \item horsepower (\texttt{hp})
  \item weight in thousands of lbs (\texttt{wt})
    \end{compactitem}
    So the model is
    \[ mpg_i = \beta_0 + \beta_1 cyl_i + \beta_2 hp_i +
      \beta_3 wt_i + \epsilon_i \]
    Now do the following.
    \begin{enumerate}
    \item Looking at the summary of the fitted model, the coefficient for weight 
      $\beta_3 \approx -3.17$. What is the interpretation of $\beta_3$?
      
    \item How do the number of cylinders and horsepower affect fuel
      efficiency?
     
    \item Plot \texttt{mpg} as a function of
    \texttt{wt}. Overlay a fitted regression line from the
    full model onto the plot.  When plotting the regression line you
    should show \texttt{mpg} at the average \texttt{cyl} and average \texttt{hp}.
    In other words, it's a two-dimensional plot, but for the other
    variables that are not shown, we compute \texttt{mpg} at their
    average values. So you want to overlay
    \[ mpg_i = \beta_0 + \beta_1 \overline{cyl} +
      \beta_2 \overline{hp} +
      \beta_3 wt_i \]
    onto the data. You can use \texttt{coef()} to extract the
    coefficients from the fitted model object.

  \item Plot the actual \texttt{mpg} vs. the predicted (fitted)
    mpg. If your fitted model is stored in an object named
    \texttt{fm}, then you can get the predicted price as follows.
    \begin{Verbatim}
      mtcars$pred <- fitted(fm)
    \end{Verbatim}
    % $
    or
    \begin{Verbatim}
      mtcars$pred <- predict(fm)
    \end{Verbatim}
    % $

  \item In the summary output of the fitted model, the estimated residual
    standard error is reported to be
    $\hat{\sigma}_{\epsilon}=2.512$. Independently compute this quantity. In
    other words, use the actual values from the data and the fitted
    values from the model to compute the residual standard error
    yourself.  The formula is
    \[ \hat{\sigma}_{\epsilon} = \sqrt{ \frac{\sum_{i=1}^n \left(y_i - \hat{y_i}\right)^2}{n-k}} \]
    where $y_i$ and $\hat{y_i}$ are the actual and fitted values of observation
    $i$, respectively, $n$ is the total number of observations, and $k$ is the
    number of fitted parameters in the model. $n-k$ is the degrees of freedom.

  \item Do you think that a linear model is appropriate for this data?
  \end{enumerate}

% i'm not yet sure what i want to do with this problem.
\item \emph{Linear regression with numeric and categorical predictors.}
  The following sales data were
  collected for one particular product from a company for the past 10
  seasons.  The data are the price of the product that the company
  itself charged, the price that its competitor charged (for the
  competitor's version of the same product), the corresponding sales
  of the company's product, and the season. This data is available
  in the file \texttt{sales.csv}.

\begin{center}
\begin{tabular}{rrrl}
company price & competitor price & sales (1000s) & season \\ \hline
         \$10.2         &     \$9.9  &71.1 &winter \\
         11.6         &     9.9  &63.0 &summer \\
          9.8         &    11.7  &71.7 &winter \\
         13.7         &     9.5  &58.3 &summer \\
         12.0         &     8.9  &61.8 &summer \\
         11.2         &    10.1  &66.0 &summer \\
         10.2         &    11.1  &71.2 &winter \\
         10.6         &    10.7  &66.9 &winter \\
          9.5         &    12.6  &72.5 &winter \\
         11.8         &    10.0  &65.4 &winter
\end{tabular}
\end{center}

\begin{enumerate}
\item Create a visualization that shows all of the data on one plot.
  One idea is to plot sales vs. price, distinguish company and competitor
  price by symbol shape, and distinguish season by color.
  \item Fit a linear regression model 
\[
y = \beta_0 + \beta_1x_1 + \beta_2x_2 + \beta_3x_3
\]
where $y$ represents the sales in 1000s of units, $x_1$ represents
the company price in dollars, $x_2$ represents the competitor price in
dollars, and $x_3$ is an indicator variable as follows
\[
x_3 = \begin{cases} 0 \quad \text{if season is winter} \\
1 \quad \text{if season is summer}
\end{cases}
\]
\item Provide an interpretation each of the fitted parameters $\beta_0$, $\beta_1$,
  $\beta_2$, and $\beta_3$. \label{aa}
\item Do you think that competitor price should be included in the
  model?  Explain the reasoning for your answer. \label{bb}
\item What is the expected company sales if the company and the
  competitor both set their prices to \$11 for the winter season?  Use
  the full model, regardless of your answer to
  part~\ref{bb} \label{cc}
\end{enumerate}

\item \emph{Adding an interaction term to a regression model.}  Refer
  to model \ref{eq:speed-height-gender} where we express the fastest
  speed ever driven for a sample of college students as a function of
  height and gender. \label{ex:interaction}

\begin{equation*}
  speed_i = \beta_0 + \beta_1 height_i + \beta_2 gender_i + \epsilon_i, \qquad i=1,\ldots,n
\tag{\ref{eq:speed-height-gender} revisited}
\end{equation*}

The summary regression output is shown in Figure~\ref{fig:regoutput2}
and the fitted model is plotted against the data in
Figure~\ref{fig:speed-height-gender}. In this model, the effect
of height on speed is the same for both genders. Now, we want to investigate the
possibility that the effect of height differs by gender. In other words,
we want to allow the slope for height to be different for males and
females (if the data indicates so). To do this, we add an interaction 
term to the model.

\begin{equation}
  speed_i = \beta_0 + \beta_1 height_i + \beta_2 gender_i + \beta_3 height_i gender_i + 
\epsilon_i, \qquad i=1,\ldots,n
\end{equation}

Remember that in the mathematical model $height_i$ and $gender_i$ are both numeric.
The term $\beta_3 height_i gender_i$ is literally the product of $beta_3$, $height_i$,
and $gender_i$. The corresponding model formula in R is:
\begin{Verbatim}[samepage=true]
speed ~ height + gender + height*gender
\end{Verbatim}

\begin{enumerate}
\item Fit this model in R. The summary output will contain an entry for the
coefficent $\beta_3$ (the coefficient on the interaction term).
\item Provide an interpretation for the effect of height on speed and also
for the effect of gender on speed. \emph{Hint:} When an interaction term
is present, we can no longer interpret the coefficients in isolation. For example,
the effect of height on speed will involve both $\beta_1$ and $\beta_3$.
\item Plot the fitted model against the data. Your plot will be
  similar to Figure~\ref{fig:speed-height-gender} except the slopes
  will differ by gender.
\item Do you think that, at least statistically, the interaction term is
  appropriate for this model?
\end{enumerate}

\item \emph{Computing $R^2$.} Refer to model~\ref{eq:reg1} and the 
output shown in Figure~\ref{fig:regoutput1}. Using the equations
\ref{eq:R2} as a guide, implement your own computation of $R^2$ for
model \ref{eq:reg1}. In other words, using R as a calculator,
compute $R^2$ on your own.

% I need to check if this problem needs to be re-written.  The data is
% from harrell's book, but we may be able to use it. the idea is to
% fit a logistic regression model with one numeric predictor and one
% categorical predictor and view the response on the probability scale
% graphically.
\item \emph{Logistic regression.} The file \texttt{harrell.csv}
contains data on 40 people. The variables are \texttt{age} in years,
\texttt{gender}, a categorical variable with two levels, \texttt{female}
and \texttt{male}, and \texttt{response}, a 0/1 indicator variable for
whether the person responded to a medical treatment (1 means that the
person responded). Fit a logistic regression model with
\texttt{response} as the dependent variable and \texttt{age} and
\texttt{gender} as the independent variables.

\begin{enumerate}
\item How does the probability of response change for a 42-year-old male
compared to a 52-year-old male?
\item Which gender has a higher probability of response to the medical
treatment?
\item What is the effect on the odds of response for a one--year
increase in age?
\item Make a plot of the probability of response as a function
of age, with one curve for females and one curve for males.
\end{enumerate}

\item \emph{Transformations on data.}
  Suppose that a response variable $u$ is related to an explanatory variable $v$ as
\[
u = \gamma_0e^{\gamma_1v}
\]
where $\gamma_0$ and $\gamma_1$ are parameters to be estimated.  Even
though the relationship between $u$ and $v$ is nonlinear, explain how
you could use simple linear regression to find estimates for
$\gamma_0$ and $\gamma_1$.

\begin{solution}
\bs
We can transform the relationship into a linear one
by taking logarithms.
\[
\text{ln}(u) = \text{ln}(\gamma_0) + \gamma_1v
\]
Then we have the form of a linear regression model with $\beta_0 = \text{ln}(\gamma_0)$
and $\beta_1 = \gamma_1$. After fitting the linear model,
\[
\hat{\gamma}_0=e^{\hat{\beta}_0}, \qquad \hat{\gamma}_1 = \hat{\beta}_1
\]
\end{solution}

\subsubsection*{Classification}

\subsubsection*{Time Series}

% re-written by braeden
\item \emph{Flattening the curve.}  The file
  \texttt{us-daily-covid-cases.csv} contains data on the daily number
  of confirmed COVID-19 cases in the United States from March 1st to
  July 7th 2020.

\begin{enumerate}
\item Construct a single plot that shows the actual number of cases,
a 7-day moving average, and exponentially smoothed average for 
the case where the smoothing parameter is $\alpha=0.3$. 
\item Briefly explain the main differences between the moving average
values and the exponential smoothing values.
\end{enumerate}

% this problem is OK.
\item \emph{Simulation of a simple trading strategy.}
  The file \texttt{BTC-USD.csv} contains one year's worth of price
  data on Bitcoin. The \texttt{Close} column is the price of
  Bitcoin in USD on \texttt{Date}. 

\begin{enumerate}
\item Read the data into an R data frame.
\item Turn the \texttt{Date} field into a proper Date variable rather
  than a character string.
\item Make a time-series plot with \texttt{Date} on the x axis and
  \texttt{Close} on the y axis.
\item Make a scatter plot with \texttt{Volume} on the x axis and
  \texttt{Close} on the y axis. Do you see a pattern? Try making
  this plot using a logarithmic transformation on \texttt{Volume} and
  \texttt{Close}.
\item Create a one-step-ahead forecast for the closing price using
  simple exponential smoothing. Use a smoothing parameter value
  of $\alpha = 0.5$.
\item Using your forecast simulate a very simple trading strategy.
  The strategy is: if the forecast is for a price increase then buy,
  otherwise if the forecast is for a price decrease then sell. Don't
  worry about the bid/ask spread, trading fees, or any of the messy
  details. You are simply buying or selling one Bitcoin at the closing
  price. Keep track of your gain/loss. How did you do? Do you
  have any criticisms of this trading strategy?
\item  Now plot your forecasted/predicted price and the actual price
  on the same plot. Do you see a general behavior in the forecast?
  Experiment with different values for $\alpha$.
\end{enumerate}
  
\end{enumerate}
%%% Local Variables:
%%% TeX-master: "main"
%%% End:
