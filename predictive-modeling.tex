\chapter{Predictive Modeling}

\section{Linear Regression}

\section{Time Series}

\section{Exercises}

\begin{enumerate}

% this problem is OK
\item \emph{Linear regression with a single predictor variable.}
In the \texttt{datasets} package in R there is a data set named
\text{faithful} that
contains data on the Old Faithful geyser in Yellowstone National Park,
Wyoming, USA.  The variables in the data set are
\begin{compactitem}[$\circ$]
\item \texttt{eruptions} the eruption time in minutes
\item \texttt{waiting} the time in minutes until the next eruption
\end{compactitem}

To view information about the data set, type
\begin{Verbatim}
> ?faithful
\end{Verbatim}
at the R prompt. Use this data to perform the following exercises.

\begin{enumerate}
\item Create histograms of \texttt{eruptions} and \texttt{waiting}
  (separately). \label{hist}

\item Create a scatter plot (i.e. an x--y plot) with \texttt{eruptions}
  on the x axis and \texttt{waiting} on the y axis. \label{scat}

\item From the histograms and the scatter plot that you created, what
  can you say about the behavior of the Old Faithful geyser? \label{interp}
  
\item Fit a linear regression model (using the \texttt{lm()} function
  with \texttt{waiting} as the response variable and \texttt{eruptions}
  as the only predictor variable. Print a summary of the results.
  \begin{enumerate}
  \item What is the interpretation of the intercept? \label{int}
  \item What is your interpretation of the fitted coefficient
    on \texttt{eruptions}? \label{dur}

  \item You just observed an eruption of duration 4 minutes.
    Make a prediction on how long you will have to wait until
    the next eruption. Can you make any statement about the
    uncertainty in your prediction? In other words, can
    you give a range for the time until the next eruption?
    Don't worry about being exact with your range, just
    give something reasonable. \label{pred}.
  \end{enumerate}
\end{enumerate}

% written by braeden
\item \emph{Using multiple linear regression to summarize a dataset.}
  For this problem we will be using a dataset called \texttt{mtcars}
  from the \texttt{datasets} package in R. This dataset contains data
  about different types of cars.  Fit a linear regression model using
  \texttt{lm()} with miles per gallon (\texttt{mpg}) as the response
  variable and the following predictor variables:
  \begin{compactitem}
  \item number of cylinders (\texttt{cyl})
  \item horsepower (\texttt{hp})
  \item weight in thousands of lbs (\texttt{wt})
    \end{compactitem}
    So the model is
    \[ mpg_i = \beta_0 + \beta_1 cyl_i + \beta_2 hp_i +
      \beta_3 wt_i + \epsilon_i \]
    Now do the following.
    \begin{enumerate}
    \item Looking at the summary of the fitted model, the coefficient for weight 
      $\beta_3 \approx -3.17$. What is the interpretation of $\beta_3$?
      
    \item If you are an engineer designing a car and you want to
      increase its fuel efficiency (\texttt{mpg}), would you want to
      increase or decrease the weight of the vehicle? What about
      number of cylinders and horsepower?
     
    \item Plot \texttt{mpg} as a function of
    \texttt{wt}. Overlay a fitted regression line from the
    full model onto the plot.  When plotting the regression line you
    should show \texttt{mpg} at the average \texttt{cyl} and average \texttt{hp}.
    In other words, it's a two-dimensional plot, but for the other
    variables that are not shown, we compute \texttt{mpg} at their
    average values. So you want to overlay
    \[ mpg_i = \beta_0 + \beta_1 \overline{cyl} +
      \beta_2 \overline{hp} +
      \beta_3 wt_i \]
    onto the data. You can use \texttt{coef()} to extract the
    coefficients from the fitted model object.

  \item Plot the actual \texttt{mpg} vs. the predicted (fitted)
    mpg. If your fitted model is stored in an object named
    \texttt{fm}, then you can get the predicted price as follows.
    \begin{Verbatim}
      mtcars$pred <- fitted(fm)
    \end{Verbatim}
    % $
    or
    \begin{Verbatim}
      mtcars$pred <- predict(fm)
    \end{Verbatim}
    % $

  \item In the summary output of the fitted model, the estimated residual
    standard error is reported to be
    $\hat{\sigma}_{\epsilon}=2.512$. Independently compute this quantity. In
    other words, use the actual values from the data and the fitted
    values from the model to compute the residual standard error
    yourself.  The formula is
    \[ \hat{\sigma}_{\epsilon} = \sqrt{ \frac{\sum_{i=1}^n \left(y_i - \hat{y_i}\right)^2}{n-k}} \]
    where $y_i$ and $\hat{y_i}$ are the actual and fitted values of observation
    $i$, respectively, $n$ is the total number of observations, and $k$ is the
    number of fitted parameters in the model. $n-k$ is the degrees of freedom.

  \item Do you think that a linear model is appropriate for this data?
  \end{enumerate}
    
\end{enumerate}
