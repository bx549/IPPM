\chapter{Decision Problems}

\section{Games Against Nature}

\section{Games Against an Opponent}

\section{Exercises}
\begin{enumerate}

% This exercise is OK
\item Rules for decision--making under ignorance (that is to
  say, decision--making without probabilities). You have the opportunity to go on a
  blind date, but you are hesitant.  You are lonely and would like to
  find the love of your life; however, you dislike awkward
  situations. Furthermore, you find it difficult to estimate the
  probability that this particular blind date will turn out to be the
  love of your life, but you know this probability is
  non-negligible. To be a little more precise, you have the following
  values: finding the love of your life is worth 1000, being in an
  awkward date situation (i.e. being on a date and knowing that you
  will not see the person again) is worth -10, and staying home
  watching Netflix is worth zero.

\begin{enumerate}
\item Formulate a decision problem for deciding whether to go on the
blind date or to stay home.
\item Use the maximin rule to solve the problem.
\item Use the minimax regret rule to solve the problem.
\end{enumerate}

\begin{solution}
\bs The decision problem can be represented with the following table.
\\[.2in]
\begin{tabular}{ccc}
\multicolumn{3}{c}{decision matrix} \\
 & find love & lots of awkward moments \\ \hline
go on date & 1000 & -10 \\
decline date & 0 & 0 
\end{tabular}
\\[.2in] 
The maximin rule tells you to decline the date because it has
the best of all the worst possible outcomes. To use minimax regret, we
form the regret matrix.  \\[.2in]
\begin{tabular}{ccc}
\multicolumn{3}{c}{regret matrix} \\
 & find love & lots of awkward moments \\ \hline
go on date & 0 & -10 \\
decline date & -1000 & 0
\end{tabular}
\\[.2in] Minimax regret tells you to go on the date because the
possibility of not finding love has the most regret.
\end{solution}


% this problem needs to be re-written. It is IMS chapter 13 problem 21.
\item A real estate investor has the opportunity to purchase land currently
zoned residential. If the county board approves a request to rezone the property
as commercial within the next year, the investor will be able to lease the
land to a large discount firm that wants to open a new store on the property.
However, if the zoning change is not approved, the investor will have to sell
the property at a loss. Profits (in thousands of dollars) are shown in the 
following payoff table.

\begin{tabular}{rcc}
& \multicolumn{2}{c}{State of Nature} \\
& rezoning approved & rezoning not approved \\
Decision Alternative & $s_1$ & $s_2$ \\
purchase & 600 & -200\\
do not purchase & 0 & 0
\end{tabular}

\begin{enumerate}
    \item If the probability that rezoning will be approved is 0.5, what
    decision is recommended? What is the expected payoff?
    \item The investor can purchase an option to buy the land. Under the option,
    the investor maintains the right to purchase the land anytime during the next
    three months while learning more about possible resistance from area residents.
    Probabilities are as follows:
    
    \begin{tabular}{rrr}
    \multicolumn{3}{c}{Let $H$ = high resistance to zoning}\\
    \multicolumn{3}{c}{Let $L$ = low resistance to zoning}\\
    $P(H)=0.55$ & $P(s_1 \mid H)=0.18$ & $P(s_2 \mid H)=0.82$ \\
    $P(L)=0.45$ & $P(s_1 \mid L)=0.89$ & $P(s_2 \mid L)=0.11$ 
    \end{tabular}
    
    What is the optimal decision strategy if the investor uses the option
    period to learn more about the resistance from area residents before making
    the purchase decision?
    \item If the option will cost the investor an additional \$\num{10000}, 
    should the investor purchase the option? Why or why not? What is the maximum
    that the investor should be willing to pay for the option?
\end{enumerate}
\begin{solution}
\bs For part a), the best decision (according to expected monetary value)
is to purchase the land. The expected profit is \$200.

For part b), If there is high resistance to rezoning (H) then
purchasing the land would yield an expected profit of -\$56 (in \$1000s).
Not purchasing the land would yield zero. If there is low
resistance to rezoning (L) then purchasing the land would yield
an expected profit of \$512 (in \$1000s). Not purchasing the
land would yield zero. So the optimal decision strategy is to
not purchase the land if H and to purchase the land if L.

For part c), the expected profit from purchasing the option
(but before actually purchasing the option) is
\[ 0 \times P(H) + 512 \times P(L) = \$230.4 \]
The maximum amount that the investor should be willing to pay is
\[ \$230.4 - \$200 = \$30.4 ~\text{(in \$1000s)} \]
\end{solution}

\end{enumerate}
